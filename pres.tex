% !TEX TS-program = xelatex
% !TEX encoding = UTF-8 Unicode
% !BIB program = biber

\documentclass[aspectratio=169]{beamer}
\usepackage[style=authoryear-icomp, mincitenames=1, maxcitenames=2, url=false, backend=biber]{biblatex}
\usepackage{amsmath, amssymb, mathtools}
\usepackage{textgreek}
\usepackage{caption}
\usepackage{hyperref}
\usepackage[absolute,overlay]{textpos}
\usepackage{appendixnumberbeamer}
\usepackage{pgfpages}
\usepackage{xpatch}
\usepackage{tabularx}

\usepackage{gfsneohellenicot}

% \usepackage{cmbright}
% \usepackage[OT1]{fontenc}

\usetheme{metropolis}
\usepackage{fontspec}
\usepackage[sfdefault]{atkinson}
% \setsansfont[BoldFont={Atkinson Hyperlegible}]{Atkinson Hyperlegible}

%--------------------------------
% DRAFTING AND SLIDE CONSTRUCTION
%--------------------------------
% \pgfpagesuselayout{4 on 1}
\setbeameroption{hide notes}
% \setbeameroption{show only notes}
% \setbeameroption{show notes on second screen=right}

%-------------------------
% PATCHING & CUSTOMISATION
%-------------------------

\definecolor{UOABlue}{HTML}{00457D}
\definecolor{EngPurple}{HTML}{4F2D7F}
\definecolor{Mustard}{HTML}{906E01}
\setbeamercolor{frametitle}{bg=UOABlue}
\setbeamercolor{palette primary}{bg=UOABlue}
\setbeamercolor{progress bar}{fg=orange}
\metroset{titleformat=smallcaps}
\metroset{sectionpage=simple}
\metroset{progressbar=frametitle}

% \makeatletter
% \setlength{\metropolis@progressinheadfoot@textwidth}{2pt}
% \makeatother

\renewcommand*{\footnotesize}{\tiny}
\renewcommand*{\bibfont}{\scriptsize}
\renewcommand*{\thefootnote}{\fnsymbol{footnote}}

\newcommand{\norm}[2][2]{\left\lVert #2 \right\rVert_{#1}}
\newcommand{\twraps}[2][2cm]{\text{\parbox{#1}{\centering #2}}}
\newcommand{\tPhi}{\text{\textbf{\textPhi}}}

\DeclareMathOperator*{\argmin}{arg\,min}
\DeclareMathOperator*{\argmax}{arg\,max}

\xapptobibmacro{cite}{\setunit{\nametitledelim}\printfield{title}}{}{}

\DeclareSymbolFont{pazosymbols}{OMS}{zplm}{m}{n}
\makeatletter
\ExplSyntaxOn %define \xmapsto
\DeclareMathSymbol \c_pazo_minus \mathbin { pazosymbols } 0
\DeclareMathSymbol \c_pazo_mapstochar \mathrel { pazosymbols } { "37 }
\DeclareMathSymbol \c_pazo_rightarrow \mathrel { pazosymbols } { "21 }
\cs_new_protected_nopar:Nn \pazo_relbar: {
  \mathrel {
    \mathpalette \mathsm@sh \c_pazo_minus
  }
}
\cs_set_protected_nopar:Nn \MT_mapsto_fill: {
  \arrowfill@ { \c_pazo_mapstochar \pazo_relbar: } \pazo_relbar: \c_pazo_rightarrow
}
\ExplSyntaxOff
\makeatother

\newenvironment{popupbox}[1]%
{%
    \metroset{block=fill}%
    \begin{textblock*}{0.9\textwidth}(0.1\textwidth,0.15\textheight)%
        \begin{alertblock}{#1}%
}%
{       \end{alertblock}%
    \end{textblock*}%
}%


% \metroset{numbering=none}

%--------------
% DOCUMENT INFO
%--------------

\title[]{Computational Methods in Epidemic Simulation and Inference}
\subtitle{Not-So-Great Models in Complex Situations}
\author{David Wu}
\date{17 August 2023}
% \titlegraphic{\vspace{4.5cm}\flushright\includegraphics[width=2cm,height=2cm]{exam_resource/uoa-v-colour2.png}} 
% \addbibresource{biblio.bib}
% \addbibresource{biblio_samoa.bib}
% \addbibresource{covid_reports.bib}
% \addbibresource{covid_others.bib}



\begin{document}

\frame[plain,noframenumbering]{\titlepage}


\begin{frame}
  \frametitle{About Me}
  % \begin{columns}
  %   \begin{column}{0.33\textwidth}
  %     \centering
  %     \includegraphics[width=0.5\textwidth]{img/uoa-v-colour2.png}

  %     PhD from Dept. Engineering Science at the University of Auckland in New Zealand
  %   \end{column}
  %   \begin{column}{0.33\textwidth}
  %     \centering
  %     \includegraphics[width=0.5\textwidth]{img/python.jpeg}

  %     Primarily use Python tooling
  %   \end{column}
  %   \begin{column}{0.33\textwidth}
  %     \centering
  %     \includegraphics[width=0.5\textwidth]{img/virus.jpg}

  %     Research is in the field of infectious diseases
  %   \end{column}
  % \end{columns}
  \centering
  \begin{tabular}{m{0.3\textwidth} m{0.3\textwidth} m{0.3\textwidth}}
    \begin{center}\includegraphics[width=0.2\textwidth]{img/uoa-v-colour2.png}\end{center}%
    &\begin{center}\includegraphics[width=0.2\textwidth]{img/python.jpeg}\end{center}%
    &\begin{center}\includegraphics[width=0.2\textwidth]{img/virus.jpg}\end{center}\\%
    \begin{center}{\scriptsize PhD from Dept. Engineering Science at the University of Auckland in New Zealand}\end{center}
    &\begin{center}{\scriptsize Primarily use Python tooling}\end{center}
    &\begin{center}{\scriptsize Research is in the field of infectious diseases}\end{center}
  \end{tabular}
\end{frame}

\begin{frame}
  \frametitle{Overview}
  \begin{enumerate}
    \item \hyperlink{fitting-ode}{Fitting ODE models with generalised profiling}
    \item \hyperlink{network-sim}{Simulating stochastic models on networks (maybe skip)}
    \item \hyperlink{surrogates}{Surrogate modelling techniques for inference}
  \end{enumerate}
\end{frame}

\section{Fitting ODE Models}\label{fitting-ode}

% \begin{frame}
%   \frametitle{Fitting ODEs: Summary}

%   Objective: Robustly fit a deterministic ODE model to epidemic data.
  
%   Challenges: 
%   \begin{itemize}
%     \item Trajectory matching can be expensive, and we want to deal with that problem.
%     \item When the model is only partially observed, we run into issues with identifiability.
%   \end{itemize}

%   Contributions:
%   \begin{itemize}
%     \item Novel interpretation of existing method in a likelihood-based formulation that allows for natural interpretation and uncertainty quantification in parameters and state.
%     \item Analysis and forecasting of a measles outbreak in Samoa.
%   \end{itemize}
% \end{frame}

\begin{frame}

  \frametitle{ODE Models}
  \only<1>{
    \centering
    \includegraphics[width=0.85\textwidth]{img/sir_diagram.png}
  }
  \only<2>{
    \begin{equation*}
      \begin{aligned}
        \frac{dS}{dt} &= -\beta \frac{SI}{N},\\
        \frac{dI}{dt} &= \beta \frac{SI}{N} - \alpha I,\\
        \frac{dR}{dt} &= \alpha I.
      \end{aligned}
    \end{equation*}
  }
  \only<3>{
    \begin{equation*}
      \frac{d}{dt}\begin{bmatrix}
        S\\I\\R
      \end{bmatrix} = \begin{bmatrix}
        -\beta\frac{SI}{N}\\
        \beta\frac{SI}{N} - \alpha I\\
        \alpha I
      \end{bmatrix}.
    \end{equation*}
  }
  \only<4>{
    \begin{equation*}
      \begin{gathered}
        \frac{d}{dt}(x) = \overbrace{f(x; \theta)}^{\text{RHS}}\\
        x = \begin{bmatrix}
          S & I & R
        \end{bmatrix}^T\\
        \theta = \begin{bmatrix}
          \beta & \alpha
        \end{bmatrix}^T\\
      \end{gathered}
    \end{equation*}
  }

\end{frame}

\begin{frame}[t]
  \frametitle{Solving ODE Models}

  Euler's Method:
  \begin{itemize}
    \item $\frac{dx}{dt} \approx \frac{x(t+\Delta t) - x(t)}{\Delta t}$
    
    \item $x(t + \Delta t) = x(t) + (\Delta t) f(x(t); \theta)$
  \end{itemize}
  \only<2>{
    \centering
    \includegraphics[height=0.7\textheight]{img/euler_method.png}
  }
  \only<3>{
  There are methods with higher-order accuracy (wrt $\Delta t$) that require evaluation of $f$ at multiple $t$.

  Adaptive methods exist that take smaller steps in regions of time where the behaviour is "faster" than expected, in order to improve accuracy.

  These methods can be very slow in particular regions of parameter space. (*Problem 1)
  }
\end{frame}


\begin{frame}[t]
  \frametitle{Fitting ODEs}

  We have some data $y$, and we want to fit an ODE model to it: find $\theta$ that would likely reconstruct $y$.

  We typically work with a simple additive error model:

  \begin{equation*}
    \begin{gathered}
      y(t) = \overbrace{g(x(t))}^{\mathclap{\text{observation model}}} + \varepsilon,\\
      \frac{d}{dt} x(t) = f(x(t); \theta),\\
      \varepsilon \sim \mathcal{N}(0, \Gamma).
    \end{gathered}
  \end{equation*}
\end{frame}

\begin{frame}[t]
  \frametitle{Fitting ODEs: Classical Methods}
  Non-linear Least Squares (frequentist)

  \begin{equation*}
    \min_{\theta, x(0)}  \left\lVert y - g\left(\int_0^{t} f(x(t);\theta)\right)  \right\rVert^2_2
  \end{equation*}
  \pause

  Bayesian Analogue

  \begin{equation*}
    p(\theta, x_0 | y) = \underbrace{\left\lVert y - g\left(\int_0^{t} f(x(t);\theta)\right) \right\rVert^2_{\Gamma^{-1}}}_{\text{likelihood: } \cdot p(y|\theta, x_0)} \;\cdot \;p(\theta, x_0)
  \end{equation*}

\end{frame}

\begin{frame}
  \frametitle{Problems}

  \begin{enumerate}
    \setlength{\itemsep}{1ex}
    \item Numerical integration can be slow.
    \item Solving the ODE implicitly assumes that the model is correct.
  \end{enumerate}
\end{frame}

\begin{frame}
  \frametitle{A Partial Solution: Collocation on Splines}

  Instead of integrating the ODE, getting a state, and then comparing against the data, we could go backwards.
  
  We could get a proposal for a vector field from the data*, then compare the vector field with the RHS of the ODE.
  % To get the proposal, we will fit a smooth (differentiable) function to the data: a spline. 


\end{frame}

% \begin{frame}
%   \frametitle{Fitting ODEs: Methods}
  
%   Generalised profiling (parameter cascading) in the likelihood framework.
%   \vspace{2em}

%   \only<1>{
%     Classical Generalised Profiling

%     Inner:
%     \begin{equation}
%       \hat{c}(\theta) = \min_c \norm{y - G(\Phi{c})}^2 + \lambda \norm{\mathcal{D}(\Phi{c}) - f(\Phi{c}, \theta)}^2
%     \end{equation}
%     Outer:
%     \begin{equation}
%       \hat\theta = \min_\theta \norm{y - G(\Phi\hat{c}(\theta))}^2.
%     \end{equation}
%   }

%   \only<2>{
%     Likelihood-based Interpretation

%     \begin{equation}
%       \hat\theta, \hat{c} = \min_{\theta, c} \log\mathcal{L}(\theta, c) = \norm[\Gamma^{-1}]{y - G(\Phi{c})}^2 + \norm[\widehat{\Sigma^{-1}}]{\mathcal{D}(\Phi{c}) - f(\Phi{c},\theta)}^2
%     \end{equation}
%   }
% \end{frame}

% \begin{frame}
%   \frametitle{Fitting ODEs: Methods}
%   Uncertainty Quantification

%   \begin{enumerate}
%     \item Likelihood Profiling
%     \item Bootstrapping with Randomised Maximum Likelihood
%   \end{enumerate}
% \end{frame}


%%%%%%-------------------------------

\begin{frame}
  \frametitle{Fitting ODEs: Measles in Samoa}
\centering
  \vspace{0.25em}
  % \includegraphics[height=0.75\textheight]{exam_resource/samoa_data.pdf}

  \vspace{-0.25em}
  {\tiny \linespread{0}Plots of the available data for the 2019 Samoan measles outbreak (current hospitalisations and cumulative discharges available, but not shown). Top-left: cumulative reported cases; top-right: cumulative reported deaths; bottom-left: report incidence (daily rate of new cases since last report); bottom-right: cumulative reported hospitalisations.\\}

\end{frame}

% \begin{frame}
%   \frametitle{Fitting ODEs: Measles in Samoa}
%   \centering
%   \includegraphics[height=0.85\textheight]{exam_resource/seirh.png}
% \end{frame}

% \begin{frame}
%   \frametitle{Fitting ODEs: Measles in Samoa}
%   \centering
%   Maximum Likelihood Estimates

%   \includegraphics[width=0.45\textwidth]{img/37_samoa/big_state.pdf}
%   \includegraphics[width=0.45\textwidth]{img/37_samoa/smol_state.pdf}
% \end{frame}

% \begin{frame}
%   \frametitle{Fitting ODEs: Measles in Samoa}
%   \centering
%   \includegraphics[height=0.85\textheight]{img/37_samoa/bivariate_profile.pdf}
% \end{frame}

% \section{Network Models}\label{network-sim}

% \begin{frame}
%   \frametitle{Network Models: Summary}

%   Objective: Incorporate heterogeneity and stochasticity into an epidemic model through a contact network, and do it at scale. Strongly driven by modelling COVID-19 in New Zealand.

%   Challenges:
%   \begin{itemize}
%     \item Large number of individuals and possible reactions -- algorithms run into memory or computational limitations
%   \end{itemize}

%   Contributions:
%   \begin{itemize}
%     \item Novel extensions to the Gillespie algorithm to allow for efficient simulation using a rejection-based method.
%     \item Numerous applications of the model for scenario modelling of COVID-19 in New Zealand.
%   \end{itemize}
% \end{frame}

% \begin{frame}
%   \centering
%   \frametitle{Network Models: COVID-19 in NZ}

%   \includegraphics[width=0.8\textwidth]{tikz/covid_timeline_nz}

%   \includegraphics[height=0.6\textheight]{exam_resource/cases_nz_stats_nz}

% \vspace{-1em}
%   {\tiny
%     Count of active cases, recovered individuals and deceased individuals due to COVID-19 in NZ (Source: Stats NZ)}
% \end{frame}


% \begin{frame}
%   \frametitle{Network Models: The Network}
%   \centering
%   Bipartite Networks: Explicitly modelling contexts of infection

%   \includegraphics[height=0.7\textheight]{img/40_covid/contact-multiple}

%   \hfill {\small *work of others in the Network Contagion Team}
% \end{frame}

% \begin{frame}
%   \frametitle{Network Models: The Model (roughly)}
%   \centering

%   \includegraphics[height=0.8\textheight]{exam_resource/transmission_diagram_big_no_distancing.pdf}
% \end{frame}

% \begin{frame}
%   \frametitle{Network Models: Methods}
%   % \centering
%   Computational Bottlenecks and Complications:
%   \begin{enumerate}[<+->]
%     \item Updating rates of neighbours upon becoming infected (exposed)
%     \item Separation of probability of (ever) firing and rate of firing
%     \item Non-Markovian inter-event/waiting times
%   \end{enumerate}

%   Solution: Rejection-based sampling
% \end{frame}

% \begin{frame}
%   \frametitle{Network Models: Methods}
%   Example: Updating rates%

%   \only<3->{Instead}%

%   \begin{center}%
%     \includegraphics<1>[height=0.7\textheight]{exam_resource/rej/inf2sus_1}%
%     \includegraphics<2>[height=0.7\textheight]{exam_resource/rej/inf2sus_2}%
%     \includegraphics<3>[height=0.7\textheight]{exam_resource/rej/rej_1}%
%     \includegraphics<4>[height=0.7\textheight]{exam_resource/rej/rej_2}%

%     \vspace{-1em}
%     \only<2>{All nodes must update rates}%
%     \only<3>{Only the infected updates}%
%     \only<4>{Upon an infection event firing, randomly sample and perform acceptance check}%
%   \end{center}%

% \end{frame}

% \begin{frame}
%   \frametitle{Network Models: Methods}
%   Non-Markovian inter-event times

%   \centering
%   \includegraphics[height=0.8\textheight]{img/40_covid/gamma_bounded.pdf}
% \end{frame}

% \begin{frame}
%   \frametitle{Network Models: Results}
%   November 2020 unknown case analysis

%   \centering
%   \includegraphics[height=0.7\textheight]{exam_resource/nov_2020_outbreak.png}
% \end{frame}

% \begin{frame}
%   \frametitle{Network Models: Results}
%   End of August 2021 (Delta) outbreak: AL4 (left) vs AL3 (right)

%   \centering
%   \includegraphics[width=0.475\textwidth]{img/40_covid/results/gp_condition/9_sept_al4.png}
%   \includegraphics[width=0.475\textwidth]{img/40_covid/results/gp_condition/9_sept_al3pess.png}

%   \hfill{\small *uncertainty quantification done by others in the Network Contagion Team}
% \end{frame}

% \section{Surrogate Models for Inference}\label{surrogates}

% \begin{frame}
%   \frametitle{Surrogate Modelling}
  
%   Objective: Enable inference of parameters of a complex (network-based) model using a surrogate ODE model.

%   Challenges:
%   \begin{itemize}
%     \item Likelihoods for a stochastic network model are intractable; likelihood-free methods are far too expensive for response planning of outbreaks.
%   \end{itemize}

%   Contribution:
%   \begin{itemize}
%     \item Proof-of-concept study on a simple SIR model.
%   \end{itemize}
% \end{frame}

% \begin{frame}
%   \frametitle{Surrogate Modelling: Methods}
%   \centering
%   \includegraphics<1>[height=0.7\textheight]{exam_resource/Z_S_C}
%   \includegraphics<2>[height=0.7\textheight]{exam_resource/Z_S_E}

%   \only<2>{where $e$ is a multivariate Gaussian}
% \end{frame}

% \begin{frame}
%   \frametitle{Surrogate Modelling: Methods}

%   $\zeta(\theta, \eta) \approx s(\theta) + e(\theta, \eta)$

%   $e(\theta, \eta) \sim \mathcal{N} (\mu(\theta, \eta), \Sigma(\theta, \eta))$

%   To estimate $e$:
%   \begin{enumerate}
%   \item Generate realisations of $\zeta$: $Z = \{\zeta(\theta_1, \eta_1), \zeta(\theta_2, \eta_2), \dots\}$

%   \item Compute $E = \{\zeta(\theta_1, \eta_1) - s(\theta_1), \dots\}$

%   \item Take statistics of $E$: 
%   \begin{itemize}
%     \item $\mu(\theta, \eta) = \frac{1}{|E|}\sum_{e_i \in E} e_i$
%     \item $\Sigma(\theta, \eta) = \frac{1}{|E|^2}\left< e_i - \mu , e_i - \mu\right>$
%   \end{itemize}
%   \end{enumerate}
% \end{frame}

% \begin{frame}
%   \frametitle{Surrogate Modelling: Proof of Concept}
%   Approximate a stochastic network SIR model with an SIR ODE:
%   \centering

%   \includegraphics[height=0.85\textheight]{img/50_surrogate/runrunvar.pdf}
% \end{frame}

% \begin{frame}
%   \frametitle{Surrogate Modelling: Proof of Concept}

%   The Curve Registration Problem

%   \centering
%   $\text{Stoch}(t; \theta, \eta) \approx \text{ODE}(t {\color{red}- \tau}; \theta) + e(\mu, \Sigma)$

%   \includegraphics<1>[height=0.8\textheight]{img/50_surrogate/runrunvar.pdf}
%   \includegraphics<2>[height=0.8\textheight]{img/50_surrogate/k100_state_figure_510}
% \end{frame}

% \begin{frame}
%   \frametitle{Surrogate Modelling: Proof of Concept}
%   A rough linear relationship on log-log between transmission parameter and $\tau$

%   \centering
%   \includegraphics[height=0.85\textheight]{img/50_surrogate/linear_linear_relationship_k}
% \end{frame}

% \begin{frame}{Surrogate Modelling: Proof of Concept}

%   \centering
%   \includegraphics[height=0.85\textheight]{img/50_surrogate/surrogate_r0rec_profile}
% \end{frame}

% \section{Summary}

% \begin{frame}
%   \frametitle{Summary}
%   \begin{enumerate}
%     \item \hyperlink{fitting-ode}{Fitting ODE models with generalised profiling}
%     \item \hyperlink{network-sim}{Simulating stochastic models on networks}
%     \item \hyperlink{surrogates}{Surrogate modelling techniques for inference}
%   \end{enumerate}
% \end{frame}

% \begin{frame}{Thanks}
%   \begin{columns}
%     \begin{column}{0.45\textwidth}
%       \includegraphics[width=\textwidth]{exam_resource/Screenshot from 2023-04-12 16-14-37.png}
%     \end{column}
%     \begin{column}{0.45\textwidth}
%       \includegraphics[width=\textwidth]{exam_resource/XbYFITdo.jpeg}
%     \end{column}
    
%   \end{columns}

% \end{frame}

\begin{frame}[standout]
  \centering
\end{frame}
% 
\end{document}
